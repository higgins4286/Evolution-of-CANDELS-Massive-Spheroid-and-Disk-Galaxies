\documentclass{article}
\usepackage[utf8]{inputenc}
\usepackage{comment}

\title{Gal Evo Annotated Bib}
\author{Lauren}
\date{Last Update September 8th 2020}

\begin{document}

\maketitle

\textbf{Next time, add in info from highlighted sections.}

\section{Barro et al 2013}

CANDELS: The Progenitors of Compact Quiescent Galaxies at z $\approx$ 2\\
https://ui.adsabs.harvard.edu/abs/2013ApJ...765..104B/abstract

\subsection{Sample}
M $> 10^{16} M_{\odot}$ \\
z = 1.4 - 3 \\
Combo of Hubble/WFC3 and two fields of CANDELS 

\subsection{Method}
They analyzed large samples of galaxies and studied their star formation rate density, size (half 
light radius), color, mass, and shape (Sersic Index). 

\subsection{Conclusion}
They wanted to see if compact star forming 
galaxies where progenitors for compact quiescent galaxies. They pose two key paths from compact, star
forming galaxies to compact quiescent galaxies. One is an early track (z $\geq$ 2) with a 
formation path of rapidly quenched cSFGs fading in to cQGs that later enlarge within the quiescent 
phase and the other is a late track (z $\leq$ 2) in which larger SFGs form extended QGs without 
passing through a compact state.

\subsection{Key Plots}
Figure 1. The sample selection of quiescent and star forming galaxies in log(sSFR) vs log(M) space and r$_{e}$ vs log(M) space. \\
\\
Figure 2. Evolution of sSFR vs. $\Sigma_{1.5} \equiv M/r_{e}^{1.5}$ \\
\textbf{What does r$_{e}^{1.5}$ represent? Is that one-half more than the radius of the galaxy?} 
\\
Figure 4. Number of SFGs, cSFGs, and QGs in the sample. \\
\\
Figure 5. Number density evolution of massive cGSs (red) and cSFGs (blue).\\
\\
Figure 6. Schematic view of a two path formation scenario for QGs.

\subsection{Connection}
This is the early work on what my project is based off. I think this is the first time this evolution path was posed.


\section{Bell et al. 2012}

What turns galaxies off? The different morphologies of star-forming and quiescent galaxies since z $\approx$ 2 from CANDELS \\
https://ui.adsabs.harvard.edu/abs/2012ApJ...753..167B/abstract \\

\subsection{Sample}
Galaxies from HST/WFC3 imaging from the CANDELS Multi-Cycle Treasury Survey with data from SDSS with stellar masses $> 3 \times 10^{10} M_{\odot}$ from z=2.2 to the present epoch, a time span of 10 Gyr.
They found the rest-frame magnitudes using their stellar population model fits and supplemented that with public 24$\mu$m data from SpUDS.

\subsection{Method}
They explore the relationship between rest-frame optical color, stellar mass, star formation activity, and galaxy structure. Specifically they studied how the ditrtibution of star formation activity, Sersic index (n), stellar mass, inferred velocity dispersion ($\sigma$), and stellar surface density ($\Sigma_{1.5}$)

\subsection{Conclusion}
They found that Sersic index is over all the best measure of quiescence over each redshift bin. Second best waS velocty dispersion and surface density. At z$<$ 0.05 Sersic index correlates the strongest. At z $>$ 0.6 velocity dispersion and surface density also correlate well with quiescence with Sersic index only marginally better. Note, each correlation has a considerable amount of scatter with many systems containing promiment disks; however, the majority of quiescent galaxies have prominent bulges.

\subsection{Key Plots}
Figure 5-8. Evolution of U-V rest-frame color (in Vega magnitudes) as a function of the key quantities Sersic index (n), stellar mass, inferred velocity dispersion ($\sigma$), and stellar surface density ($\Sigma_{1.5}$), in six different redshift bins. \\
\\
Figure 9. Quiescent fraction in three broad redshift bins as a function of the rank of n, $\Sigma$, $\sigma'$, and $M_{*}$. Where rank denotes where a galaxy in the sorted list of the n, $\sigma'$, and $M_{*}$ in the sample at that redshift of interest.\\
\\
Figure 11 \& 12. F160W postage stamps of quiescent and star forming galaxies (respectively) with n$<$2 and n$>$3 and stellar surface densities between 10$^{9}$ and 10$^{10}$ $M_{\odot}$ kpc$^-2$.

\subsection{Connection}
This paper shows the key quantities that I will be organizing as I sift through the data. It also shows how morphology and quenching is connected to these key quantities. These connection will fuel my work after I get the data organized.

\section{van der Wel et al 2014}

\subsection{Sample}
Spectroscopic and photometric redshifts, stellar mass estimates, and rest-frame colors from the 3D-HST survey combines with structural parameter measurements from CANDELS imaging in 0 $<$ x $<$ 3.

\subsection{Method}


\subsection{Conclusion}
They found that early-type galaxies are on average smaller than late-type galaxies at all
redshifts and they found a significantly different rate of average size evolution at 
fixed galaxy mass, with fast evolution for the early-type population, $R_{eff} \propto 
(1+z)^{-1.48}$, and moderate evolution for the late-type population, $R_{eff} \propto 
(1+z)^{-0.75}$. Furthermore, at all redshifts the size-mass relation is shallow, $R_{eff}
\propto M^{0.22}_{*}$, for late-type galaxies with stellar mass $> 3 \times 10^{9} 
M_{\odot}$, and steep, $R_{eff} \propto M^{0.75}_{*}$, for early-type galaxies with 
stellar masses $> 2 \times 10^{10} M_{\odot}$ The intrinsic scatter is $\leq$0.2 dex for 
all galaxy types and redshifts. For late-type galaxies, the logarithmic size distribution
is not symmetric but is skewed toward small sizes: at all redshifts and masses, a tail of
small late-type galaxies exists that overlaps in size with the early-type galaxy 
population. The number density of massive ($\approx10^{11} M_{\odot}$), compact ($R_{eff} < 2 
kpc$) early-type galaxies increases from z = 3 to z = 1.5–2 and then strongly decreases 
at later cosmic times. 

\subsection{Key Plots}
Figure 3. Rest-frame U-V color vs. stellar mass in six redshift bins.\\
\\
Figure 6. Paramerterized redshift evolution of the size-mass relation, from the power law model fits shown in Figure 5 (Figure 5 references Figure 2).\\
\\
Figure 7. Evolution-corrected average sizes at $M_{*} = 5 \times 10^{10} M_{\odot}$ for late-type galaxies (top panel, in blue) and early-type galaxies (bottom panel, in red). (hubble parameter might provide a more accurate description).\\
\\
Figure 10. Sixe distributions for early- and late-type galaxies as function of stellar mass and redshift.\\
\\
Figure 12. Size evolution of galaxies in a narrow (0.3 dex) mass bin around 10$^{10} M_{\odot}$ \\
\\
Figure 13. Number density evolution of compact early-type galaxies.\\

\subsection{Key Tables}
Table 1. Results form the Parameterized Fits to the Size-Mass Distribution from the form 
$R_{eff}/kpc = A(M_{*}/5 \times 10^{10} M_{\odot})^{\alpha}$

\subsection{Connection}
This characterizes galaxy evolution in two separate morphologies based on effective size. They used several paramaters that I will see in the database and I see how important they are in characterising galaxy evolution.

\section{Kartaltepe et al 2012}
\subsection{Sample}
Deep 100 and 160 $\mu$m observations in GOODS-South from GOODS-\textit{Herschel} combined with $HST$/WFC3 NIR imaging from CANDELS.

\subsection{Method}
They visually classified FIR selected sample of 52 ultraluminous IR galaxies (ULIRGs; L$_{IR} > 10^{12} L_{\odot}$ at z $\sim$ 2. They compared these galaxies to galaxies at the same redshift and $H$-band magnitude distribution with lower IR luminosities. They then compared the fractions found in this sample with a z $\sim$ 1 sample from GOODS-H and COSMOS across a wide luminosity range. Then, they investigated the position of the z $\sim$ 2 ULIRGs, along with 70 z $\sim$ 2 LIRGs, on the specific star formation rate vs redshift plane.

\subsection{Conclusion}
These two samples indicate that the fractions of objects with disk and spheroid morphologies are roughly the same but there are significantly more mergers, interactions, and irregular galaxies among the ULIRGs (72$_{-7}^{+5}\%$ vs 32$\pm 3\% $). The combination of morphologies indicate that early-state interactions, minor-mergers, and disk instabilities could play an important role in ULRIGs at at z $\sim$ 2. 

They found that the fraction of disks in the waie range of luminosities sample at z ~ 1 decreases systematically with L$_{IR}$ while the fraction of mergers and interactions increases as has been observed locally. At comparable luminosities, the fraction of ULIRGs with various morphological classifications is similar at z $\sim$ 2 and z $\sim$ 1 with slightly fewer mergers and slightly more disks at higher redshifts.

When comparing the position of the z $\sim$ 2 ULIRGs, along with 70 z $\sim$ 2 LIRGs, on the specific star formation rate vs redshift plane, they found 52 systems to be starbursts (ie. they lie more than a factor of three above the main-sequence relation). Many of these systems are clear interactions and mergers ($\sim$ 50) compared to only 24$\%$ systems on the main sequence relation. Including irregular disks as potential minor mergers, they find that that up to $\sim$ 73$\%$ of starbursts are involved in a merger or interaction at some level.

Lastly, the final coalescence of a major merger may not be requred for the high lum of ULIRGs at z $\sim$ 2 as in the case locally, the large fraction (50$\%$-73$\%$) of interactions at all stages and potential minor mergers suggests that these processes contribute significantly to the high star formation rates of ULIRGs at $\sim$ 2.

\subsection{Key Plots}


\subsection{Key Tables}

\end{document}

\begin{comment}
\section{van der Wel et al 2014}
\subsection{Sample}
\subsection{Method}
\subsection{Conclusion}
\subsection{Key Plots}
\subsection{Key Tables}
\end{comment}

\subsection{Connection}